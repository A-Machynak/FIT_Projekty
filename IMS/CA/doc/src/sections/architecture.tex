\section{Architektura}

\subsection{Modely}
Implementace modelu elementárního CA je triviální a již byla zmíněna v předchozí
části.
Náročnější je vytvoření prostředí, ve kterém je možné efektivně experimentovat
s různými parametry. Proto byly přidány některé vhodné metody:

\begin{itemize}
	\item Zadávání pravidel jako seznam čísel přes CLI (\textit{Command Line Interface})
	\item Zadávání počátečního stavu jako seznam čísel přes CLI s možností
	zadání náhodného stavu aj.
	\item Způsob iterování pravidel ("RuleStrategy"):
	\begin{itemize}
		\item Náhodný - pro každou generaci je vybráno náhodné pravidlo
		\item Náhodný z datasetu - je vybráno náhodné pravidlo ze seznamu
		pravidel zadaných uživatelem
		\item Cyklický - je vybráno následující pravidlo ze seznamu pravidel
		zadaných uživatelem, přičemž u posledního pravidla
		je proveden návrat k prvnímu pravidlu
		\item Cyklický bez návratu - místo návratu k prvnímu pravidlu se
		opakovaně aplikuje poslední pravidlo
	\end{itemize}
\end{itemize}

Kód je dostupný ve třídě CA::Elementary (\verb|src/model/elementary.*|).


\subsection{Simulace a vizualizace}
Podstatnou částí je vizualizace CA. Tu zajišťuje velmi malý "framework",
který byl vytvořen pro potřeby zobrazení CA ve 2D.
Jedná se pouze o okno, které obsahuje plochu o velikosti zadané uživatelem.
Jednotlivé pixely je možné na této ploše poté upravovat.
Použití tohoto frameworku vypadá následovně:
\begin{enumerate}
	\item Vytvoření nového okna o zadané velikosti (\textit{Gui::PixelWindow})
	\item Nastavení časového kroku
	\item Implementace rozhraní \textit{Gui::IStepper}:
	\begin{enumerate}
		\item Inicializace - voláno pouze jednou na začátku
		\item Kroková funkce - voláno každý časový krok
		\item Podmínka ukončení - kontrolováno po každém časovém kroku
		\item Reset - voláno po zmáčknutí tlačítka; sémantika může
		být teoreticky různá, ale očekává se návrat do počátečního stavu
	\end{enumerate}
	\item Spuštění simulace s parametrem instance implementovaného rozhraní
	(v případě elementárního CA \\\textit{ModelWrapper::ElementaryWrapper})
\end{enumerate}
Tento framework je možné jednoduše použít obecně pro vizualizaci jakéhokoliv 1D/2D CA
(teoreticky i více, ale pro tyto potřeby nebyl vytvořen a bylo by to tedy komplikované).

Kód je dostupný ve třídě Gui::PixelWindow (\verb|visualization/gui/gui.*|)
a ModelWrapper::ElementaryWrapper (\verb|visualization/models_wrap/elementary_ca_wrapper.*|).

\subsubsection{Použití programu}

Příklady použití (CLI):
\begin{verbatim}
	$ ./CA_Visualization -h
\end{verbatim}

Pravidla se perodicky střídají, pro 1.
generaci zvolen 1 náhodný bod,
zvolená pravidla 105 a 106,
vypnutí opengl (obrázek představující výsledky generací je
uložen do souboru \verb|rule_105_106.png| po dokončení simulace):
\begin{verbatim}
	$ ./CA_Visualization -out rule_105_106 -gloff -caargs dring rng1 105,106
\end{verbatim}

Pravidla se náhodně střídají, počáteční stav (0, 0, 1, 0, 0, 1, ...),
zvolená pravidla 15, 22, 30, 60.
\begin{verbatim}
	$ ./CA_Visualization -caargs drandom 0,0,1 15,22,30,60
\end{verbatim}
